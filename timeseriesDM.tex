\section{The Time Series Data Model Summary}
The different components of the Time Series model are summarized in Table~\ref{table:tsmodel}. Each of these components is an independent part and metadata associated to each specific case case should be included in the VOTable when applicable. 
\begin{landscape}
\begin{table}
\begin{center}
\begin{tabular}{|p{0.5\textheight}|p{0.2\textheight}|p{0.4\textheight}|p{0.2\textheight}|p{0.1\textheight}|p{0.15\textheight}|}
%\sptablerule
\hline
\celcol{Utype }  & \celcol{UCD1+} & \celcol{Meaning} & \celcol{Default value} & \celcol{Data type} & \celcol{Required}\\
\sptablerule
\multicolumn{6}{c}{\celcol{Time axis (TIMESYS)}}\\
ts:TimeSys.timescale   &                & Time scale                                               & & string & must \\
\hline
ts:TimeSys.refposition &                & Time reference position                                  & & string & must \\
\hline
ts:TimeSys.origin      &                & Time origin of the time series given as a Julian Date    & & double & must \\
\hline
\multicolumn{6}{c}{\celcol{Case 1: photometric axis (FILTERSYS)}} \\
\hline
ts:PhotCal.identifier  & meta.ref.ivorn & Unique identifier of the Photometry Calibration instance & & string & should \\
\hline
ts:PhotCal.zeroPoint.flux.value & phot.flux.density & flux value at Zero point associated to this filter & & double & should \\ 
\hline 
ts:PhotCal.magnitudeSystem.type & meta.code & Type of magnitude system & VEGAMag & string & should \\
\hline 
ts:MeanRefMag                   &  & Mean magnitude used for differential magnitude calculation in this filter &  & double & should \\
\hline
\multicolumn{6}{c}{\celcol{Case 2: position (COOSYS) }} \\
\multicolumn{6}{c}{\celcol{Case 3: radial velocity axis }} \\
\label{table:tsmodel}
\end{tabular}
\caption{Model components and associated UTYPES, UCD1+ of the Time Series Annotation}
\end{center}
\end{table}
\end{landscape}
