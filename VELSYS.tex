\subsection{\elem{VELSYS} Element}
\elem{VELSYS} contains information on the velocity system of the observations. A service providing time series of the type radial-velocity-curve \textbf{MUST} provide this element. To reference the velocity system defined by a \elem{VELSYS} element, \elem{FIELD}s (and possibly \elem{PARAM}s) \textbf{MUST} reference the \elem{VELSYS} using the VOTable \attr{ref} attribute. A \elem{VELSYS} element referenced via a \attr{ref} attribute \textbf{SHOULD} appear before the element that references it. The attributes of \elem{VELSYS} are the following: 

\begin{description}
     \item[\elem{ID}] This attribute is used to reference the \elem{VELSYS} elements from the elements using the photometric system. This attribute is mandatory. 
     \item[\elem{velocity}] This attribute defines the type of value. Allowed values are \verb|VELOCITY| and \verb|REDSHIFT|
     \item[\elem{definition}] This attribute allows to translate redshift to doppler Velocity \cite{STC}. Allowed values are \verb|OPTICAL|, \verb|RADIO| and \verb|RELATIVISTIC|. \todo{check FITS}
     \item[\elem{refPosition}] This attribute is the reference position. The values SHOULD be taken from the IVOA \emph{refposition} vocabulary (\url{http://www.ivoa.net/rdf/refposition}). The other possible values are \verb|LSR|, \verb|LSRD|, \verb|GALACTOCENTRIC| \verb|LOCAL_GROUP_CENTER|. This attribute is mandatory.
\end{description}

\noindent
\begingroup\footnotesize
\begin{tcolorbox}
\begin{verbatim}
<VELSYS ID="velocity_sys" velocity="VELOCITY" 
           definition="OPTICAL" refPOSITION="LSR" />
\end{verbatim}
\end{tcolorbox}
\endgroup

%\subsection{\elem{ORBITSYS} Element}
%Not sure we should do this at all... I mean this is other kind of stuff which comes from an in-deepth analysis. 
%Semi major axis, Eccentricity, Inclination, Longitude of ascending node, Argument of periapsis, Epoch T of mean anomaly M (if present) or periapsis (if M is not present)
%The following parameters may be provided: Mean anomaly M: 0 ≤ M < 360˚; if M is present, T will be considered to
%represent the epoch of M Orbital period P taken from page 26 / 109 from STC https://arxiv.org/pdf/1110.0504.pdf

%\subsection{Adding images, spectra or other files}
%If we want to associate the images or spectra used to derive the \elem{FIELD} as a pointer adding type=location value, meaning that the field contains a way to access the data instead of the data itself. Using the \elem{LINK} partial URI can be defined and referenced in a column. The full URI can be generated by appendint the content of the cell to the content of the \elem{href} element. 

%\textbf{<TD encoding="base64"> STREAM element for improving efficiency}
