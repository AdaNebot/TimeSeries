\subsection{\elem{GROUP}}
\elem{GROUP} contains the time series data itself. A service providing time series \textbf{MUST} provide this element.

The \elem{GROUP} is identified with:
\begin{description}
\item ucd=''timeseries'' \todo{Please give input here}
\item utype =''timeseries'' \todo{Please give input here}
\end{description}

The \elem{GROUP} contains:
\begin{description}
     \item[\elem{name}] The name of the group. This attribute is mandatory and it should be set to \verb|timeseries|. 
     \item[\elem{DESCRIPTION}] Human readable text describing the time series. This element is mandatory. 
     \item[\elem{FIELDref}] One or several \elem{FIELDref} (or \elem{PARAMref}) to reference to the \elem{TIMESYS} element. At least one such reference MUST be provided.
     \item[\elem{FIELDref}] One or several \elem{FIELDref} (or \elem{PARAMref}) to reference to the \elem{COOSYS}, \elem{FILTERSYS} as applicable. 
     \item One or several PARAM or PARAMref. This will describe the BAND for instance. While PARAM is usually defining valid for a whole table in this case it will be valid only for the GROUP.  
     \item[\elem{SELECT}] One SELECT attribute. This element provides a way to reference rows instead of columns. This attribute is particular important for cases in which the magnitudes given in a column are not corresponding to the same band and the name of the band is given in a different column. \todo{I hope this is clear, case of ZTF for instance.}
\end{description}

Let us consider the simplest time series case possible: a table containing only three columns \emph{time, magnitude and magnitude error}, the minimum GROUP would be:

\begingroup
\begin{verbatim}
<TIMESYS ID="htime" refposition="HELIOCENTER" timeorigin="JD-origin" 
           timescale="UNKOWN"/>
<TIMESYS ID="ttime" refposition="TOPOCENTER" timeorigin="MJD-origin" 
           timescale="UNKOWN"/>
<FILTERSYS ID="phot_sys_zg" uniqueIdentifier="Palomar/ZTF.g/Vega" 
           zeroPointFlux="3963.97" magnitudeSystem="Vega" 
           effectiveWavelength="4722.74" />
<FILTERSYS ID="phot_sys_zr" uniqueIdentifier="Palomar/ZTF.r/Vega" 
           zeroPointFlux="3042.0" magnitudeSystem="Vega" 
           effectiveWavelength="6339.6" />
<FILTERSYS ID="phot_sys_zi" uniqueIdentifier="Palomar/ZTF.i/Vega" 
           zeroPointFlux="2426.1" magnitudeSystem="Vega" 
           effectiveWavelength="7886.1" />
<GROUP name="timeseries_zg" utype="timeseries" > 
  <DESCRIPTION> This is a light-curve in filter zg</DESCRIPTION>
  <FIELDref ref="timesys"/>
  <FIELDref ref="magnitude_error"/>
  <FIELDref ref="filter_name"/>
</GROUP>
<GROUP name="timeseries_rg" utype="timeseries" > 
  <DESCRIPTION> This is a light-curve in several filters</DESCRIPTION>
  <FIELDref ref="timesys"/>
  <FIELDref ref="magnitude_error"/>
  <FIELDref ref="filter_name"/>
</GROUP>
<GROUP name="timeseries_ig" utype="timeseries" > 
  <DESCRIPTION> This is a light-curve in several filters</DESCRIPTION>
  <FIELDref ref="timesys"/>
  <FIELDref ref="magnitude_error"/>
  <FIELDref ref="filter_name"/>
</GROUP>
\end{verbatim}

\endgroup

In this example we have assumened some metadata for \elem{TIMESYS} and \elem{FILTERSYS} and have included to show a full example case. While for the time and the magnitude reference implies an earlier definition of the elements \elem{TIMESYS} and \elem{FILTERSYS} this is not the case for the error. The error in the magnitude references the name of the column which will be repeated in the \elem{FIELD} element directly of the \elem{TABLE}.

We could argue that the \elem{GROUP} is not needed in this case, but the idea is to have a common way to representing time series, including the more complicated cases. 

\newpage 
Let us consider now consider a more complicated case, in which the time series also includes a column to state the name of the filter for wich the magnitudes are given, i.~e. a table containing four columns \emph{time, magnitude and magnitude error, filter name}. In those cases this complicates the annotation and the need of being able to reference rows (as well as columns) becomes clear.

\begingroup
\begin{verbatim}
<?xml version='1.0'?>
<VOTABLE version="1.4" xmlns="http://www.ivoa.net/xml/VOTable/v1.3">
<RESOURCE>
<RESOURCE>
<TIMESYS ID="timesys-1" refposition="HELIOCENTER" timeorigin="JD-origin" 
           timescale="UNKOWN"/>
<TIMESYS ID="timesys-2" refposition="TOPOCENTER" timeorigin="MJD-origin" 
           timescale="UNKOWN"/>
<FILTERSYS ID="phot_sys_zg" uniqueIdentifier="Palomar/ZTF.g/Vega" 
           zeroPointFlux="3963.97" magnitudeSystem="Vega" 
           effectiveWavelength="4722.74" />
<FILTERSYS ID="phot_sys_zr" uniqueIdentifier="Palomar/ZTF.r/Vega" 
           zeroPointFlux="3042.0" magnitudeSystem="Vega" 
           effectiveWavelength="6339.6" />
</RESOURCE>
<TABLE name="timeseries_zr">
  <FIELD datatype="double" name="hjd" ucd="time.epoch;meta.main" unit="d" ref="timesys-1" >
    <DESCRIPTION>Heliocentric Julian date</DESCRIPTION>
  </FIELD>
  <FIELD datatype="double" name="mjd" ucd="time.epoch;obs.exposure" unit="d" ref="timesys-2">
    <DESCRIPTION>Modified Julian date</DESCRIPTION>
  </FIELD>
  <FIELD datatype="float" name="mag" ucd="phot.mag;em.opt" unit="mag" ref="phot_sys_zg">
    <DESCRIPTION>Magnitude</DESCRIPTION>
  </FIELD>
  <FIELD datatype="float" name="magerr" ucd="stat.error;phot.mag;em.opt" 
    unit="mag" ref="phot_sys_zg">
    <DESCRIPTION>Uncertainty in mag measurement</DESCRIPTION>
  </FIELD>
  <DATA>
  <TABLEDATA>
    ...
  </TABLEDATA>
  </DATA>
</TABLE>
<TABLE name="timeseries_zr">
  <FIELD datatype="double" name="hjd" ucd="time.epoch;meta.main" unit="d" ref="timesys-1" >
    <DESCRIPTION>Heliocentric Julian date</DESCRIPTION>
  </FIELD>
  <FIELD datatype="double" name="mjd" ucd="time.epoch;obs.exposure" unit="d" ref="timesys-2">
    <DESCRIPTION>Modified Julian date</DESCRIPTION>
  </FIELD>
  <FIELD datatype="float" name="mag" ucd="phot.mag;em.opt" unit="mag" ref="phot_sys_zr">
    <DESCRIPTION>Magnitude</DESCRIPTION>
  </FIELD>
  <FIELD datatype="float" name="magerr" ucd="stat.error;phot.mag;em.opt" 
    unit="mag" ref="phot_sys_zr">
    <DESCRIPTION>Uncertainty in mag measurement</DESCRIPTION>
  </FIELD>
  <DATA>
    <TABLEDATA>
    ...
    </TABLEDATA>
  </DATA>
</TABLE>
</RESOURCE>
</VOTABLE>
\end{verbatim}

\endgroup

To avoid referencing columns, which would make things a bit more complicated, we propose to create different \elem{TABLE} elements, one for each filter. This, although reduntant in the definition of some of the columns, will ease combining timeseries from the client side. 

%Alternative 2: Use the GROUP name=key thing and a different table to define the relation between elements. This 
%Alternative 3: Another possible solution would be to create a new XMLDATA serialization which would allow to tag rows, as proposed in Appendix A.8 of \cite{VOTable1.4}. But wouldn't that break the interoperability? 

