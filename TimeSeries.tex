\documentclass[11pt,a4paper]{ivoa}
\input tthdefs

\usepackage{todonotes}
\usepackage{enumitem}

\usepackage{listings}
\lstloadlanguages{XML,sh}
\lstset{flexiblecolumns=true,tagstyle=\ttfamily,
showstringspaces=False}

\usepackage{multirow}

\definecolor{LightBlue}{rgb}{0.0,0.318,0.612}

% Added this comand for my text to be added in blue and boldface

\newcommand\ada[1]{\textcolor{blue}{\textbf{#1}}}
\newcommand\elem[1]{\textcolor{LightBlue}{{\tt#1}}}
%\def\attrval#1#2{{\tt{\fg{DarkRed}#1}="{\fg{DarkPurple}#2}"}}
%\def\elem#1{{\tt{\fg{DarkRed}#1}}}

\title{Simple Cone Search}
\ivoagroup{Data Model}
\author[http://www.ivoa.net/twiki/bin/view/IVOA/AdaNebot]{Ada Nebot}
\editor{Ada Nebot}

\begin{document}
\begin{abstract}
  This document describes a proposal to annotate in a VOTable time series data. It is limitted to the most common type of time series, i.e. tabular data containing a parameter measured as a function of time. The annotation reuses elements of exisiting Data Models when possible and defines a set of new elements. This document can be taken as a test case of the more general purpose model CADM. 
\end{abstract}

\section*{Acknowledgments}
Should we acknowledge previous work from M. Graham and others

%ASTERICS

ESCAPE

\section*{Conformance-related definitions}

The words ``MUST'', ``SHALL'', ``SHOULD'', ``MAY'', ``RECOMMENDED'', and
``OPTIONAL'' (in upper or lower case) used in this document are to be
interpreted as described in IETF standard RFC2119 \citep{std:RFC2119}.

The \emph{Virtual Observatory (VO)} is a
general term for a collection of federated resources that can be used
to conduct astronomical research, education, and outreach.
The \href{http://www.ivoa.net}{International
Virtual Observatory Alliance (IVOA)} is a global
collaboration of separately funded projects to develop standards and
infrastructure that enable VO applications.


\section{Introduction}
This document describes how data providers can publish time series of tabular data to the Virtual Observatory. This document does not describe how data providers store or manipulate the data, it only concerns how the data are exposed to the world using well-defined metadata. This document assumes that the data provider already destributes time series as a single table, this table can contain several columns. This document does not limit the number of columns the time series can contain to those explecitely mentioned here. It only suggest metadata to the most common columns in time series.  

% Notes don't need the architecture diagram

\section{The Time Series Structure}
\label{elem:TIMESERIES}
This document proposes the minimum metadata in VOTables to describe time series. We assume that the tabular data can contain multiple rows for each astronomical source and for mixed types of time series (e.~g. different filters) and sometimes different types of data (e.~g. spectra, or images)

We propose the following elements to describe the elements of the time series: 

\subsection{\elem{TIMESYS} Element}
\elem{TIMESYS} contains information on the time scale, reference position and offset regarding the time of the observation \cite{VOTable1.4, TIMESYS}. A service providing time series \textbf{MUST} provide this element. To reference the time system defined by a \elem{TIMESYS} element, \elem{FIELD}s (and possibly \elem{PARAM}s) \textbf{MUST} reference the \elem{TIMESYS} using the VOTable \elem{ref} attribute. The elements of TIMESYS are those reported in the specification of VOTable1.4 \ref{VOTable1.4}. This fields are the following: 

\begin{description}
   \item[\elem{ID}] This attribute is used to reference \elem{TIMESYS} elements from the elements using the time system.
   \item[\elem{timeorigin}] This is the time origin of the time coordinate, given as a Julian Date for the the time scale and reference point
defined.  It is usually given as a floating point literal; for convenience, the magic strings \verb|MJD-origin| (standing
for 2400000.5) and \verb|JD-origin| (standing for 0) are also allowed. The timeorigin attribute \textbf{MUST} be given unless the time's representation contains a year of a calendar era, in which case it \textbf{MUST NOT} be present. In VOTables, these representations currently are Gregorian calendar years with \elem{xtype}{timestamp}, or years in the Julian or Besselian calendar when a column has \verb|yr|, \verb|a| or \verb|Ba| as its unit and no time origin is given. 
   \item[\elem{timescale}] This is the time scale used. Values \textbf{SHOULD} be taken from the IVOA \emph{timescale} vocabulary (\url{http://www.ivoa.net/rdf/timescale}). This attribute is mandatory.
   \item[\elem{refposition}] The reference position again is a simple string, the values of which SHOULD be taken from the IVOA \emph{refposition} vocabulary (\url{http://www.ivoa.net/rdf/refposition}). This attribute is mandatory.
\end{description}
  
\subsection{\elem{FILTERSYS} Element}
\elem{FILTERSYS} contains information on the photometric system of the observations. This element is partially asociated to the intermediate class PhotCal of the Photometry Data Model \cite{PhotometryDM}. A service providing time series of the type light-curve (e.~g. magnitudes, fluxes) \textbf{MUST} provide this element. To reference the photometric system defined by a \elem{FILTERSYS} element, \elem{FIELD}s (and possibly \elem{PARAM}s) \textbf{MUST} reference the \elem{FILTERSYS} using the VOTable \elem{ref} attribute. 
\begin{description}
     \item[\elem{ID}] This attribute is used to reference the \elem{FILTERSYS} elements from the elements using the photometric system. This attribute is mandatory. 
     \item[\elem{uniqueIdentifier}] This is the field to uniquely identify the zero point assigned to a filter and to a specific photometric system. We recommend the syntax: Facility/Subcategory/Band/Photometric System Type[/Suffix]\todo{I am not sure about that suffix, I just copied from the PhotDM, but this needs to be checked, please Jesus, can you have a look at that? } (e.g. SDSS/SDSS.G/G/AB, GAIA/GAIA.G/G/AB, Palomar/ZTF.r/AB). 
     \item[\elem{zeroPointFlux}] Photometric zero point associated to this instance. 
     \item[\elem{magnitudeSystem}] Magnitude system associated to this instance.
     \item[\elem{effectiveWavelength}] Effective wavelength of the filter in this instance. \todo{It could have been the mean wavelength too. This might need some discussion}
     \item[\elem{meanReference}] Mean value of the magnitude (or flux) of the reference used for calculating diffential photometry in the specified photometric system. This attribute is optional. \todo{Added this to be able to compare differential magnitudes, but this needs discussion. I'm not sure data providers would provide that, but for now I leave it.} 
\end{description}
\todo{Jesus: I would like to somehow encourage data providers to give their full info for services such as the SVO profile. But I think this are the minimal requirements.}


\subsection{\elem{GROUP}}
\elem{GROUP} contains the time series data itself. A service providing time series \textbf{must} provide this element. 
Each \elem{GROUP} contains:
\begin{description}
     \item[\elem{DESCRIPTION}] Human readable text describing the time series. This element is mandatory. 
     \item[\elem{FIELDref}] One or several FIELDref to reference to the \elem{TIMESYS} element and the \elem{FILTERSYS} when applicable. 
     \item One or several PARAM or PARAMref. This will describe the BAND for instance. While PARAM is usually defining valid for a whole table in this case it will be valid only for the GROUP.  
     \item One SELECT element (when needed). This element provides a way to reference rows instead of columns. 
\end{description}

%\section{Architecture}
%We consider the general case of a service returning time series in the form of tabular data.

\bibliography{ivoatex/ivoabib,ivoatex/docrepo,TimeSeries}


\end{document}
