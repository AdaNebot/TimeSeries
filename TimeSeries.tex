\documentclass[11pt,a4paper]{ivoa}
\input tthdefs

\usepackage{todonotes}
\usepackage{enumitem}
\usepackage{pdflscape}
\usepackage{xcolor,colortbl}
\usepackage[most]{tcolorbox}
 
\usepackage{listings}
\lstloadlanguages{XML,sh}
\lstset{flexiblecolumns=true,tagstyle=\ttfamily,
showstringspaces=False,basicstyle=\footnotesize}

\usepackage{multirow}

\definecolor{LightBlue}{rgb}{0.0,0.318,0.612}
\definecolor{LighterBlue}{rgb}{0.39,0.7,1}

% Added this comand for my text to be added in blue and boldface

\newcommand\ada[1]{\textcolor{blue}{\textbf{#1}}}

%\newcommand\elem[1]{\textcolor{LightBlue}{{\tt#1}}}
\newcommand\celcol[1]{\cellcolor{LighterBlue}{\textbf{#1}}}
%\def\attr#1{{\tt{\fg{DarkRed}#1}}}

\let\A=\href
\def\Aref#1{section~\ref{#1}}
\def\Arefs#1{section~\ref{#1}}
\def\Arefx#1{appendix~\ref{#1}}
\def\Tref#1{Table~\ref{#1}}
\def\Fref#1{Figure~\ref{#1}}
\let\fg=\color
%\topmargin=-1cm
%\raggedbottom
%\oddsidemargin=-0.8cm
%\evensidemargin=-0.8cm
%\textwidth=17.5cm
%\textheight=23.5cm
\parindent=0pt
\arrayrulewidth=0.75pt\renewcommand{\arraystretch}{1.2}
\definecolor{DarkRed}{rgb}{0.5,0,0}
\definecolor{DarkBlue}{rgb}{0,0,0.5}
\definecolor{DarkPurple}{rgb}{0.3,0.1,0.5}
\definecolor{DarkGoldenrod}{rgb}{0.72,0.5,0.05}
\def\slash {{\fg{blue}/}}
\def\attr#1{{\tt{\fg{DarkRed}#1}}}
\def\requiredattr#1{{\tt\bf{\fg{DarkBlue}#1}}}
\def\elem#1{{\tt{\fg{DarkRed}#1}}}
\def\attrval#1#2{{\tt{\fg{DarkRed}#1}="{\fg{DarkPurple}#2}"}}
\def\elemdef#1#2{{\tt\fg{blue}<}{\tt{\fg{DarkRed}#1}#2}{\tt\fg{blue}>}}
\def\literalvalue#1{{\tt"}{{\fg{DarkPurple}#1}}{\tt"}}
\def\order{$\oplus$ }
\def\unorder{{\large $\circ$ }}
\def\deprecated {$\dagger$ }
\def\choice{{$\mapsto$ }}
\newenvironment{plain}{\begin{quote}}{\end{quote}}


\title{Time Series: Annotation of light curves in VOTable}
\ivoagroup{Not Applicable}
\author[http://www.ivoa.net/twiki/bin/view/IVOA/AdaNebot]{Ada Nebot}
\author[http://www.ivoa.net/twiki/bin/view/IVOA/FrancoisBonnarel]{Francois Bonnarel}
\author[http://www.ivoa.net/twiki/bin/view/IVOA/MireilleLouys]{Mireille Louys}
\author[http://www.ivoa.net/twiki/bin/view/IVOA/LaurentMichel]{Laurent Michel}
\author[http://www.ivoa.net/twiki/bin/view/IVOA/DaveMorris]{Dave Morris}
\author[http://www.ivoa.net/twiki/bin/view/IVOA/JesusSalgado]{Jesus Salgado}
\editor{Ada Nebot}

\begin{document}
\begin{abstract}
  This document describes a proposal to annotate in a VOTable time series data. It is limited to the most common type of time series in astronomy: light curves, but it can be extended to other type of time series data easily (e.~g. radial velocities and positions). The annotation reuses elements of existing Data Models when possible and defines a set of new elements. This document can be taken as a test case of a more general purpose model. 
\end{abstract}

\section*{Acknowledgments}
This work has been supported by the ESCAPE project (the European Science Cluster of Astronomy \& Particle Physics ESFRI Research Infrastructures) that has received funding from the European Union’s Horizon 2020 research and innovation programme under the Grant Agreement n. 824064.

\section*{Conformance-related definitions}

The words ``MUST'', ``SHALL'', ``SHOULD'', ``MAY'', ``RECOMMENDED'', and
``OPTIONAL'' (in upper or lower case) used in this document are to be
interpreted as described in IETF standard RFC2119 \citet{std:RFC2119}.

The \emph{Virtual Observatory (VO)} is a
general term for a collection of federated resources that can be used
to conduct astronomical research, education, and outreach.
The \href{http://www.ivoa.net}{International
Virtual Observatory Alliance (IVOA)} is a global
collaboration of separately funded projects to develop standards and
infrastructure that enable VO applications.


\section{Introduction}
This document describes how data providers can publish time series in tabular form to the Virtual Observatory by prescribing metadata for a minimal set of columns required to describe a (photometric) time series.  It does not describe how data providers store or manipulate the data, it is only concerned with how the data is exposed to the world using well-defined metadata. We assume that the data provider already distributes time series as a single table, where this table may contain several columns containing photometry, and it may contain columns not explicitly mentioned here. 

% Notes don't need the architecture diagram
\section{Science use cases for Time Series}
In this section we describe a number of science use cases the proposed annotation of time series in this document will enable. We divide the science cases into different groups according to their common requirements. 
%What operations should clients be able to perform based on this without further user intervention?
\subsection{Case A: Light curves in same photometric band}
Common requirement: Combine photometry and light curves of a given object/list of objects in the same photometric band. 

\subsubsection{Use Case 1: Supernova classification using the light curve}
The visual light curves of the different supernova types vary in shape and amplitude, based on the underlying mechanisms of the explosion, the way that visible radiation is produced, and the transparency of the ejected material.

\subsubsection{Use Case 2: Long-term analysis of variable stars}
Long-term period and light curve variations, based on the historical data and new observations. The continuous period increase can be interpreted as a mass transfer from the secondary to the primary star. The most likely explanation of a sinusoidal variation would be a light travel time effect due to the existence of a circumbinary object as in the BX Dra triple system.

\subsubsection{Use Case 3: Discovering brown dwarfs by analyzing binary microlensing events}
Microlensing events show characteristic light curve variations which can be used to determine the type of system producing it. %(based on Shin et al. 2012)

%\subsubsection{Use Case 4: Eclipsing binary systems - phase}
% Based on G\'omez Maqueo et al 2009)

\subsection{Case B: light curves in different photometric bands}
Common requirement: Combine photometry and light curves of a given object/list of objects in different photometric bands.

\subsubsection{Use Case 4: Follow-up characterisation of supernovae}
%based on Zhang et al. arXiv:1208.6078v1)
Light curves at different wavelength provide different information allowing a better understanding of the physical processes related to the supernovae explosion.

\section{The Time Series Structure}
\label{elem:TIMESERIES}
This document proposes the minimal metadata required in VOTables to support our science cases. We assume that the tabular data can contain multiple rows for each astronomical source and for mixed types of time series (e.g., different filters, positions, radial velocities) and sometimes even include different types of data (e.g., spectra, or images). 

<<<<<<< HEAD
We propose using some elements previously defined, such as \elem{TIMESYS} and \elem{COOSYS} (when applicable) in VOTable1.4 \citet{std:VOTABLE1.4} and define a set of new elements as well as how to structure these elements. The aim is to be able to combine multiple time series when they have common elements as described in this document. We define a \elem{GROUP} element called \texttt{PHOTCAL} following the same type of structure. If these elements were to be added or modified in future versions of VOTable it should be straight forward to update this document. This note is inspired by a previous annotation strategy developed for annotating photometric data in VOTables \citep{note:seb2010-1}.
=======
We propose using some elements previously defined, such as \elem{TIMESYS} and \elem{COOSYS} (when applicable) in VOTable starting from version~1.4 \citep{2019ivoa.spec.1021O} and define a set of new elements as well as how to structure these elements. The aim is to be able to combine multiple time series when they have common elements as described in this document. We define a \elem{GROUP} element called \elem{PHOTCAL} following the same type of structure. If these elements were to be added or modified in future versions of VOTable it should be straight forward to update this document. This note is inspired by a previous annotation strategy developed for annotating photometric data in VOTables \citep{note:seb2010-1}.
>>>>>>> upstream/master

%\input{timesys.tex}
%\input{coosys.tex}

<<<<<<< HEAD
\subsection{The \texttt{PHOTCAL} Group}
The \texttt{PHOTCAL} group contains information on the photometric system of the observations. This element is partially asociated to the intermediate class PhotCal of the Photometry Data Model \citep{std:PDM}. A service providing time series of the type light-curve (e.g., magnitudes, fluxes) \textbf{MUST} provide this element. To reference the photometric system defined by a \texttt{PHOTCAL} \elem{GROUP}, \elem{FIELD}s (and possibly \elem{PARAM}s) \textbf{MUST} reference the group using the VOTable \attr{ref} attribute. A \texttt{PHOTCAL} group referenced via a \attr{ref} attribute \textbf{SHOULD} appear before the element that references it. 
=======
\subsection{\elem{PHOTCAL} Element}
\elem{PHOTCAL} contains information on the photometric system of the observations. This element is partially asociated to the intermediate class PhotCal of the Photometry Data Model \citep{2013ivoa.spec.1005S}. A service providing time series of the type light-curve (e.~g. magnitudes, fluxes) \textbf{MUST} provide this element. To reference the photometric system defined by a \elem{PHOTCAL} element, \elem{FIELD}s (and possibly \elem{PARAM}s) \textbf{MUST} reference the \elem{PHOTCAL} using the VOTable \attr{ref} attribute. A \elem{PHOTCAL} element referenced via a \attr{ref} attribute \textbf{SHOULD} appear before the element that references it. 
>>>>>>> upstream/master

Each photometric calibration is defined as a \elem{GROUP} with the following mandatory terms:
\begin{description}
<<<<<<< HEAD
     \item[\attrval{name}{PHOTCAL}] The name of the \elem{GROUP} \textbf{MUST} be set to \verb|PHOTCAL|. We realize that although this proposed usage of the \attr{name} is not common, it is not forbidden by VOTable \cite[][see Section 3.2]{std:VOTABLE1.4}, and in this context it will help clients interpret the contents of the \elem{GROUP}. 
     \item[\attr{ID}] This attribute is used to reference the \texttt{PHOTCAL} \elem{GROUP} from the elements using the photometric system. This attribute needs to be unique within the document so that it can be referenced by \elem{FIELD}s and is mandatory.
     \item[\attr{DESCRIPTION}] This attribute is used to describe the \texttt{PHOTCAL} \elem{GROUP}. This attribute is optional, but recommended.
     \item[\attrval{utype}{timeseries:PhotometryPoint}]\todo{Something ought to be here, if nothing else the relation of this to the fixed name.} \dots
=======
     \item[\attrval{name}{PHOTCAL}] The name of the \elem{GROUP} \textbf{MUST} be set to \verb|PHOTCAL|. We realize that although this proposed usage of the \attr{name} is not common, it is not forbidden by VOTable (\citet{2019ivoa.spec.1021O}, section 3.2), and in this context it will help clients interpret the contents of the \elem{GROUP}. 
     \item[\attr{ID}] This attribute is used to reference the \elem{PHOTCAL} \elem{GROUP} from the elements using the photometric system. This attribute needs to be unique within the document so that it can be referenced by \elem{FIELD}s and is mandatory.
     \item[\attr{DESCRIPTION}] This attribute is used to describe the \elem{PHOTCAL} \elem{GROUP}. This attribute is optional, but recommended.
     \item[\attrval{utype}{timeseries:PhotometryPoint}]
>>>>>>> upstream/master
\end{description}

The attributes of this \elem{GROUP} are defined as \elem{PARAM}s containing the following information: 
\begin{description}
\item[\elem{filterIdentifier}] This is the \elem{PARAM} element to uniquely identify the zero point assigned to a filter and to a specific photometric system. This element MUST have the following attributes:
\begin{description}
    \item[\attrval{name}{filterIdentifier}]
    \item[\attrval{ucd}{meta.id;instr.filter}]
    \item[\attrval{utype}{photDM:PhotometryFilter.identifier}]
    \item[\attrval{datatype}{char}]
    \item[\attrval{value}{value}] Value SHOULD follow the syntax: Facility/Subcategory/Band (e.~g. \attrval{value}{GAIA/GAIA.G/G}, \attrval{value}{Palomar/ZTF.r}). 
\end{description}
\item[\elem{zeroPointFlux}] Photometric zero point associated to this instance. This element MUST have the following attributes:
\begin{description}
    \item[\attrval{name}{zeroPointFlux}]
    \item[\attrval{ucd}{phot.mag;arith.zp}]
    \item[\attrval{utype}{photDM:PhotCal.zeroPoint.flux.value}]
    \item[\attrval{datatype}{datatype}]
    \item[\attrval{unit}{unit}]
    \item[\attrval{value}{value}]
\end{description}
\item[\elem{magnitudeSystem}] Magnitude system associated to this instance (Vega, AB, \dots). This element MUST have the following attributes:
\begin{description}
    \item[\attrval{name}{magnitudeSystem}]
    \item[\attrval{ucd}{meta.code}] 
    \item[\attrval{utype}{photDM:PhotCal.magnitudeSystem.type}]
    \item[\attrval{datatype}{char}]
    \item[\attrval{unit}{unit}]
    \item[\attrval{value}{value}]
\end{description}
\item[\elem{effectiveWavelength}] Effective wavelength of the filter in this instance. This element MUST have the following attributes:
\begin{description}
    \item[\attrval{name}{effectiveWavelength}]
    \item[\attrval{ucd}{em.wl.effective}]
    \item[\attrval{utype}{photDM:PhotometryFilter.spectralLocation.value}] 
    \item[\attrval{datatype}{float}]
    \item[\attrval{unit}{unit}]
    \item[\attrval{value}{value}]
\end{description}
\end{description}

Proposed annotation example for a \texttt{PHOTCAL} \elem{GROUP}: 
\lstinputlisting[language=XML]{photcal_PARAM.xml}


While following the same approach as for \elem{COOSYS} and \elem{TIMESYS} would result in a more compact annotation, we decided to define this element as a \elem{GROUP} to allow for some flexibility on the data provider side. We are aware that this implies clients to work around more complex annotations. We hope to alleviate that by following the exact \attr{name} and \attr{utype} convention as proposed in this document.

The alternative to doing this, and which would simplify the annotation, is to bring this \elem{GROUP} element closer to the \elem{COOSYS} and \elem{TIMESYS} elements by fixing the units of the \attr{zeroPointFlux} and the \attr{effectiveWavelength} to Jy and Angstrom respectively and add it to future versions of VOTable. This would look as follows: 

\lstinputlisting[language=XML]{photcal_FILTERSYS.xml}


%\subsection{\elem{VELSYS} Element}
\elem{VELSYS} contains information on the velocity system of the observations. A service providing time series of the type radial-velocity-curve \textbf{MUST} provide this element. To reference the velocity system defined by a \elem{VELSYS} element, \elem{FIELD}s (and possibly \elem{PARAM}s) \textbf{MUST} reference the \elem{VELSYS} using the VOTable \attr{ref} attribute. A \elem{VELSYS} element referenced via a \attr{ref} attribute \textbf{SHOULD} appear before the element that references it. The attributes of \elem{VELSYS} are the following: 

\begin{description}
     \item[\elem{ID}] This attribute is used to reference the \elem{VELSYS} elements from the elements using the photometric system. This attribute is mandatory. 
     \item[\elem{velocity}] This attribute defines the type of value. Allowed values are \verb|VELOCITY| and \verb|REDSHIFT|
     \item[\elem{definition}] This attribute allows to translate redshift to doppler Velocity \cite{STC}. Allowed values are \verb|OPTICAL|, \verb|RADIO| and \verb|RELATIVISTIC|. \todo{check FITS}
     \item[\elem{refPosition}] This attribute is the reference position. The values SHOULD be taken from the IVOA \emph{refposition} vocabulary (\url{http://www.ivoa.net/rdf/refposition}). The other possible values are \verb|LSR|, \verb|LSRD|, \verb|GALACTOCENTRIC| \verb|LOCAL_GROUP_CENTER|. This attribute is mandatory.
\end{description}

\noindent
\begingroup\footnotesize
\begin{tcolorbox}
\begin{verbatim}
<VELSYS ID="velocity_sys" velocity="VELOCITY" 
           definition="OPTICAL" refPOSITION="LSR" />
\end{verbatim}
\end{tcolorbox}
\endgroup

%\subsection{\elem{ORBITSYS} Element}
%Not sure we should do this at all... I mean this is other kind of stuff which comes from an in-deepth analysis. 
%Semi major axis, Eccentricity, Inclination, Longitude of ascending node, Argument of periapsis, Epoch T of mean anomaly M (if present) or periapsis (if M is not present)
%The following parameters may be provided: Mean anomaly M: 0 ≤ M < 360˚; if M is present, T will be considered to
%represent the epoch of M Orbital period P taken from page 26 / 109 from STC https://arxiv.org/pdf/1110.0504.pdf

%\subsection{Adding images, spectra or other files}
%If we want to associate the images or spectra used to derive the \elem{FIELD} as a pointer adding type=location value, meaning that the field contains a way to access the data instead of the data itself. Using the \elem{LINK} partial URI can be defined and referenced in a column. The full URI can be generated by appendint the content of the cell to the content of the \elem{href} element. 

%\textbf{<TD encoding="base64"> STREAM element for improving efficiency}


For types of time series data not explicitly defined in this document we recommend to use the appropriate UCD1+ \citep{2005ivoa.spec.0819D}. The data provider is responsible for providing accurate metadata for the elements here described.


%\subsection{\elem{GROUP}}
\elem{GROUP} contains the time series data itself. A service providing time series \textbf{MUST} provide this element.

The \elem{GROUP} is identified with:
\begin{description}
\item ucd=''timeseries'' \todo{Please give input here}
\item utype =''timeseries'' \todo{Please give input here}
\end{description}

The \elem{GROUP} contains:
\begin{description}
     \item[\elem{name}] The name of the group. This attribute is mandatory and it should be set to \verb|timeseries|. 
     \item[\elem{DESCRIPTION}] Human readable text describing the time series. This element is mandatory. 
     \item[\elem{FIELDref}] One or several \elem{FIELDref} (or \elem{PARAMref}) to reference to the \elem{TIMESYS} element. At least one such reference MUST be provided.
     \item[\elem{FIELDref}] One or several \elem{FIELDref} (or \elem{PARAMref}) to reference to the \elem{COOSYS}, \elem{FILTERSYS} as applicable. 
     \item One or several PARAM or PARAMref. This will describe the BAND for instance. While PARAM is usually defining valid for a whole table in this case it will be valid only for the GROUP.  
     \item[\elem{SELECT}] One SELECT attribute. This element provides a way to reference rows instead of columns. This attribute is particular important for cases in which the magnitudes given in a column are not corresponding to the same band and the name of the band is given in a different column. \todo{I hope this is clear, case of ZTF for instance.}
\end{description}

Let us consider the simplest time series case possible: a table containing only three columns \emph{time, magnitude and magnitude error}, the minimum GROUP would be:

\begingroup
\begin{verbatim}
<TIMESYS ID="htime" refposition="HELIOCENTER" timeorigin="JD-origin" 
           timescale="UNKOWN"/>
<TIMESYS ID="ttime" refposition="TOPOCENTER" timeorigin="MJD-origin" 
           timescale="UNKOWN"/>
<FILTERSYS ID="phot_sys_zg" uniqueIdentifier="Palomar/ZTF.g/Vega" 
           zeroPointFlux="3963.97" magnitudeSystem="Vega" 
           effectiveWavelength="4722.74" />
<FILTERSYS ID="phot_sys_zr" uniqueIdentifier="Palomar/ZTF.r/Vega" 
           zeroPointFlux="3042.0" magnitudeSystem="Vega" 
           effectiveWavelength="6339.6" />
<FILTERSYS ID="phot_sys_zi" uniqueIdentifier="Palomar/ZTF.i/Vega" 
           zeroPointFlux="2426.1" magnitudeSystem="Vega" 
           effectiveWavelength="7886.1" />
<GROUP name="timeseries_zg" utype="timeseries" > 
  <DESCRIPTION> This is a light-curve in filter zg</DESCRIPTION>
  <FIELDref ref="timesys"/>
  <FIELDref ref="magnitude_error"/>
  <FIELDref ref="filter_name"/>
</GROUP>
<GROUP name="timeseries_rg" utype="timeseries" > 
  <DESCRIPTION> This is a light-curve in several filters</DESCRIPTION>
  <FIELDref ref="timesys"/>
  <FIELDref ref="magnitude_error"/>
  <FIELDref ref="filter_name"/>
</GROUP>
<GROUP name="timeseries_ig" utype="timeseries" > 
  <DESCRIPTION> This is a light-curve in several filters</DESCRIPTION>
  <FIELDref ref="timesys"/>
  <FIELDref ref="magnitude_error"/>
  <FIELDref ref="filter_name"/>
</GROUP>
\end{verbatim}

\endgroup

In this example we have assumened some metadata for \elem{TIMESYS} and \elem{FILTERSYS} and have included to show a full example case. While for the time and the magnitude reference implies an earlier definition of the elements \elem{TIMESYS} and \elem{FILTERSYS} this is not the case for the error. The error in the magnitude references the name of the column which will be repeated in the \elem{FIELD} element directly of the \elem{TABLE}.

We could argue that the \elem{GROUP} is not needed in this case, but the idea is to have a common way to representing time series, including the more complicated cases. 

\newpage 
Let us consider now consider a more complicated case, in which the time series also includes a column to state the name of the filter for wich the magnitudes are given, i.~e. a table containing four columns \emph{time, magnitude and magnitude error, filter name}. In those cases this complicates the annotation and the need of being able to reference rows (as well as columns) becomes clear.

\begingroup
\begin{verbatim}
<?xml version='1.0'?>
<VOTABLE version="1.4" xmlns="http://www.ivoa.net/xml/VOTable/v1.3">
<RESOURCE>
<RESOURCE>
<TIMESYS ID="timesys-1" refposition="HELIOCENTER" timeorigin="JD-origin" 
           timescale="UNKOWN"/>
<TIMESYS ID="timesys-2" refposition="TOPOCENTER" timeorigin="MJD-origin" 
           timescale="UNKOWN"/>
<FILTERSYS ID="phot_sys_zg" uniqueIdentifier="Palomar/ZTF.g/Vega" 
           zeroPointFlux="3963.97" magnitudeSystem="Vega" 
           effectiveWavelength="4722.74" />
<FILTERSYS ID="phot_sys_zr" uniqueIdentifier="Palomar/ZTF.r/Vega" 
           zeroPointFlux="3042.0" magnitudeSystem="Vega" 
           effectiveWavelength="6339.6" />
</RESOURCE>
<TABLE name="timeseries_zr">
  <FIELD datatype="double" name="hjd" ucd="time.epoch;meta.main" unit="d" ref="timesys-1" >
    <DESCRIPTION>Heliocentric Julian date</DESCRIPTION>
  </FIELD>
  <FIELD datatype="double" name="mjd" ucd="time.epoch;obs.exposure" unit="d" ref="timesys-2">
    <DESCRIPTION>Modified Julian date</DESCRIPTION>
  </FIELD>
  <FIELD datatype="float" name="mag" ucd="phot.mag;em.opt" unit="mag" ref="phot_sys_zg">
    <DESCRIPTION>Magnitude</DESCRIPTION>
  </FIELD>
  <FIELD datatype="float" name="magerr" ucd="stat.error;phot.mag;em.opt" 
    unit="mag" ref="phot_sys_zg">
    <DESCRIPTION>Uncertainty in mag measurement</DESCRIPTION>
  </FIELD>
  <DATA>
  <TABLEDATA>
    ...
  </TABLEDATA>
  </DATA>
</TABLE>
<TABLE name="timeseries_zr">
  <FIELD datatype="double" name="hjd" ucd="time.epoch;meta.main" unit="d" ref="timesys-1" >
    <DESCRIPTION>Heliocentric Julian date</DESCRIPTION>
  </FIELD>
  <FIELD datatype="double" name="mjd" ucd="time.epoch;obs.exposure" unit="d" ref="timesys-2">
    <DESCRIPTION>Modified Julian date</DESCRIPTION>
  </FIELD>
  <FIELD datatype="float" name="mag" ucd="phot.mag;em.opt" unit="mag" ref="phot_sys_zr">
    <DESCRIPTION>Magnitude</DESCRIPTION>
  </FIELD>
  <FIELD datatype="float" name="magerr" ucd="stat.error;phot.mag;em.opt" 
    unit="mag" ref="phot_sys_zr">
    <DESCRIPTION>Uncertainty in mag measurement</DESCRIPTION>
  </FIELD>
  <DATA>
    <TABLEDATA>
    ...
    </TABLEDATA>
  </DATA>
</TABLE>
</RESOURCE>
</VOTABLE>
\end{verbatim}

\endgroup

To avoid referencing columns, which would make things a bit more complicated, we propose to create different \elem{TABLE} elements, one for each filter. This, although reduntant in the definition of some of the columns, will ease combining timeseries from the client side. 

%Alternative 2: Use the GROUP name=key thing and a different table to define the relation between elements. This 
%Alternative 3: Another possible solution would be to create a new XMLDATA serialization which would allow to tag rows, as proposed in Appendix A.8 of \cite{VOTable1.4}. But wouldn't that break the interoperability? 


\subsection{\elem{TABLE}}
%\todo{We could propose to define one \elem{TABLE} element to describe the content of each timeseries with common metadata (e.~g. one per filter) with no \elem{DATA}, and to add in a separate \elem{TABLE} element the \elem{DATA} part wich will referece the approprate \elem{TABLE} element. But to be honest I am not sure about the pros and contras of that at the moment of writing.} 

We propose to use the \elem{TABLE} element for describing the contents of the time series data itself. One \elem{TABLE} element MUST be present for each \elem{PHOTCAL} \elem{GROUP} element defined. We also propose to annotate the \elem{TABLE} as a timeseries using \elem{PARAM}s defined as elements from Obscore \citep{2017ivoa.spec.0509L}:

The \elem{TABLE} must include:

\begin{itemize}
\item A human-readable \attr{DESCRIPTION} that ideally conveys enough
human-readable information that basic scientific use of the time series
is possible without referring to literature.

\item A \elem{PARAM} describing the obscore dataproduct type with the 
following mandatory attributes:
     \begin{compactitem}
        \item \attrval{name}{dataproduct\_type}
        \item \attrval{ucd}{meta.code.class}
        \item \attrval{utype}{obscore:ObsDataset.dataProductType}
        \item \attrval{datatype}{char}
        \item \attrval{arraysize}{"*"}
        \item \attrval{value}{timeseries}
     \end{compactitem}

\item A \elem{PARAM} describing the dataproduct subtype with the
following mandatory attributes:

     \begin{compactitem}
        \item \attrval{name}{dataproduct\_subtype}
        \item \attrval{ucd}{meta.code.class}
        \item \attrval{utype}{obscore:ObsDataset.dataProductSubtype}
        \item \attrval{datatype}{char}
        \item \attrval{arraysize}{"*"}
     \end{compactitem}

The parameter's \xmlel{value} attribute should be set to
\texttt{lightcurve} for time series following this note, including the
photometric part.  Other values, for instance for radial velocity
curves, could be used in the future; a suitable controlled vocabulary
would be created at that point.

     \item One or several columns referencing a \elem{TIMESYS} element.
     Clients can use any such column as the independent variable of the
     time series.

     \item One or more columns referencing a \texttt{PHOTCAL}
     \elem{GROUP}.  Clients can use any such column as a dependent
     variable of a time series.  Non-photometric time series will
     replace this mechanism some some other requirement.
\end{itemize}

\lstinputlisting[language=XML]{table_TIMESERIES.xml}

We explored other ways of annotating the data, for instance defining different tables and the relation between them using foreignKey as possible according to Section 4.10 of VOTable \citep{2019ivoa.spec.1021O}, or even proposing a new XMLDATA serialization. After exploring those possibilities we conclude that the proposed solution is the most appropriate within the actual and most accepted usage of VOTablefor both existing data providers and clients.

\section{Examples Serializations}
\subsection{Case 1: lightcurves in one filter}
Let us consider the simplest time series case possible: a table containing only three columns:
\begin{center}
   \emph{time, magnitude, magnitude error}
\end{center}

The minimum VOTable would look like this:

\lstinputlisting[language=XML]{vot-ex1-GROUP.xml}

This annotation would also work in the case of tables including multiple light filters in the form
\begin{center}
   \emph{time\_1, magnitude\_1,magnitude\_error\_1, time\_2, magnitude\_2,magnitude\_error\_2}
\end{center}

\subsection{Case 2: lightcurves in multiple filters}
Let us consider now consider a more complicated case, in which the time series includes a column with name of the photometric filter of each row of the time series, i.~e. a table containing four columns:
\begin{center}
   \emph{time, magnitude, magnitude error, filter name}.
\end{center}
In those cases this complicates the annotation and the need of being able to reference rows (as well as columns) becomes an issue. It is to overcome such problem, that we propose to use different \elem{TABLE} elements for different types of products. We propose therefore to annotate a VOTable as shown in the following example: 

\lstinputlisting[language=XML]{vot-ex2-GROUP.xml} 

%To avoid referencing columns, which would make things a bit more complicated, we propose to create different \elem{TABLE} elements, one for each filter. This, although reduntant in the definition of some of the columns, will ease combining timeseries from the client side. 

%Alternative 2: Use the GROUP name=key thing and a different table to define the relation between elements. This 
%Alternative 3: Another possible solution would be to create a new XMLDATA serialization which would allow to tag rows, as proposed in Appendix A.8 of \citet{2019ivoa.spec.1021O}. But wouldn't that break the interoperability? 


%\todo{Do we have to say something about claiming a new capability in the registry? }
%\todo{Do we want to add a summary table with utype, UCD1+, Meaning, Default value, Data type, required?} 

%\section{The Time Series Data Model Summary}
The different components of the Time Series model are summarized in Table~\ref{table:tsmodel}. Each of these components is an independent part and metadata associated to each specific case case should be included in the VOTable when applicable. 
\begin{landscape}
\begin{table}
\begin{center}
\begin{tabular}{|p{0.5\textheight}|p{0.2\textheight}|p{0.4\textheight}|p{0.2\textheight}|p{0.1\textheight}|p{0.15\textheight}|}
%\sptablerule
\hline
\celcol{Utype }  & \celcol{UCD1+} & \celcol{Meaning} & \celcol{Default value} & \celcol{Data type} & \celcol{Required}\\
\sptablerule
\multicolumn{6}{c}{\celcol{Time axis (TIMESYS)}}\\
ts:TimeSys.timescale   &                & Time scale                                               & & string & must \\
\hline
ts:TimeSys.refposition &                & Time reference position                                  & & string & must \\
\hline
ts:TimeSys.origin      &                & Time origin of the time series given as a Julian Date    & & double & must \\
\hline
\multicolumn{6}{c}{\celcol{Case 1: photometric axis (FILTERSYS)}} \\
\hline
ts:PhotCal.identifier  & meta.ref.ivorn & Unique identifier of the Photometry Calibration instance & & string & should \\
\hline
ts:PhotCal.zeroPoint.flux.value & phot.flux.density & flux value at Zero point associated to this filter & & double & should \\ 
\hline 
ts:PhotCal.magnitudeSystem.type & meta.code & Type of magnitude system & VEGAMag & string & should \\
\hline 
ts:MeanRefMag                   &  & Mean magnitude used for differential magnitude calculation in this filter &  & double & should \\
\hline
\multicolumn{6}{c}{\celcol{Case 2: position (COOSYS) }} \\
\multicolumn{6}{c}{\celcol{Case 3: radial velocity axis }} \\
\label{table:tsmodel}
\end{tabular}
\caption{Model components and associated UTYPES, UCD1+ of the Time Series Annotation}
\end{center}
\end{table}
\end{landscape}

\bibliography{ivoatex/ivoabib,ivoatex/docrepo}

\end{document}
